\documentclass[master=wit,english]{kulemt}
\setup{% Verwijder de "%" op de volgende lijn bij UTF-8 karakterencodering
  %inputenc=utf8,
  title={Avoiding local minima in Deep Learning: a nonlinear optimal control approach},
  author={Jan Scheers},
  promotor={Prof.\,dr.\,ir.\ Panos Patrinos},
  assessor={Prof.\,dr.\,ir.\ Dirk Nuyens\\Prof.\,dr.\,ir.\ Johan Suykens},
  assistant={ir.\ Brecht Evens}}
% Verwijder de "%" op de volgende lijn als je de kaft wil afdrukken
%\setup{coverpageonly}
% Verwijder de "%" op de volgende lijn als je enkel de eerste pagina's wil
% afdrukken en de rest bv. via Word aanmaken.
%\setup{frontpagesonly}n

% Kies de fonts voor de gewone tekst, bv. Latin Modern
\setup{font=lm}

% Hier kun je dan nog andere pakketten laden of eigen definities voorzien

\usepackage{subcaption}
\usepackage{graphicx,amsmath,amssymb}
\usepackage{placeins}
\usepackage[ruled,vlined]{algorithm2e}
\usepackage{multirow}

% Tenslotte wordt hyperref gebruikt voor pdf bestanden.
% Dit mag verwijderd worden voor de af te drukken versie.
\usepackage[pdfusetitle,colorlinks,plainpages=false]{hyperref}

%%%%%%%
% Om wat tekst te genereren wordt hier het lipsum pakket gebruikt.
% Bij een echte masterproef heb je dit natuurlijk nooit nodig!
%\IfFileExists{lipsum.sty}%
% {\usepackage{lipsum}\setlipsumdefault{11-13}}%
% {\newcommand{\lipsum}[1][11-13]{\par Hier komt wat tekst: lipsum ##1.\par}}
%%%%%%%

%\includeonly{chap-n}
\begin{document}

\begin{preface}
  I would like to thank everybody who kept me busy the last year,
  especially my promoter and my assistant. I would also like to thank the
  jury for reading the text. My sincere gratitude also goes to my friends and my family.
\end{preface}

\tableofcontents*


\begin{abstract}
Current optimization algorithms for training deep neural network models are based on gradient descent methods, which use the backpropagation algorithm to calculate the gradient. These established algorithms have a long history and work well in practice, but still have some pitfalls such as "bad" local minima, where the stationary point found by the algorithm has worse performance than the global minimum, or also the "vanishing/exploding gradient" problem. Most improvements to these algorithms have focused on refining the gradient descent algorithm such as by adding a momentum factor, or by using Nesterov acceleration. Some other strategies aim to improve the initialization or to use "early stopping" to avoid overfitting.

In this masterthesis, the focus is on the formulation of neural network training in the context of optimal control theory. This is a perspective which allows one to view backpropagation as a "single shooting" method, but also gives the opportunity to propose new solution methods using "multiple shooting". In control theory this approach often works well for highly nonlinear problems. 

The success of the backpropagation algorithm means few attempts have been made to propose an alternative. In this thesis the "multiple shooting" method is investigated. A recently proposed inexact augmented Lagrangian framework is selected to evaluate the feasibility of solving neural network training as an optimal control problem using "multiple shooting". The Jacobian matrix at the heart of the method is analytically derived and verified numerically. The novel algorithm is then compared to the ADaptive Moment Estimation (ADAM) optimizer which is a widely used industry standard backpropagation algorithm.

For small problems the novel algorithm shows a quicker running time compared to ADAM, with similar performance. A regression problem which is difficult to train for traditional gradient descent algorithms due to the "dying ReLU" problem, converges more often using the novel algorithm. When training a network with ReLU activation function, the ADAM algorithm converges in 14 out of 60 runs, while the novel algorithm converges in 47 out of 60 runs.  However its running time and memory requirements scale quadratically with the size of the data set. In a time series prediction problem with a larger dataset, the Santa Fe laser experiment, the new method converges, but takes 20min per run compared to 20s for ADAM.

\end{abstract}

\chapter*{Samenvatting}
Huidige optimalisatiealgoritmes voor het trainen van diepe neurale netwerken zijn gebaseerd op gradi\"ent-afdalingsmethoden, die gebruik maken van het backpropagation algoritme om de gradi\"ent te berekenen. Deze algoritmes worden alom gebruikt en hebben een lange geschiedenis, maar hebben nog altijd enkele valkuilen zoals het probleem van "slechte" locale minima, waar het stationair punt gevonden door het algoritme slechter veel slechter presteert dan het globale minimum, of ook het "verdwijnende/ontploffende gradi\"ent" probleem. Andere strategi\"en stelden voor om de initialisatie te verbeteren, of om "vervroegd stoppen" aan te wenden om overfitting te vermijden.

Het trainen van een neuraal netwerk is een optimalisatieprobleem dat kan geherformuleerd worden in een context van optimale regeltechniek. Dit is een nieuw perspectief dat toelaat om backpropagation als een "single shooting" methode te zien, maar ook de gelegenheid geeft om een nieuwe oplossingsmethode voor te stellen op basis van "multiple shooting". In de theorie van regeltechniek wordt deze methode vaak aangewend voor het oplossen van zeer niet-lineaire problemen.

Het success van het backpropagation algoritme heeft er voor gezorgd dat weinig pogingen zijn gedaan om een alternatief te zoeken. In deze masterproef wordt de "multiple shooting" methode uitgewerkt. Er wordt een methode van de geaugmenteerde Lagrangiaan ge\"implementeerd om de haalbaarheid te onderzoeken van het trainen van neurale netwerken als een optimaal sturingsprobleem. Het nieuw algoritme wordt dan vergeleken met standaard backpropagation-optimalisatiealgoritmes die in de praktijk worden gebruikt.

Voor kleine problemen kan het nieuw algoritme concurreren zowel in termen van snelheid als qualiteit van oplossing. Voor een benaderingsprobleem dat traditioneel moeilijk trainbaar is met huidige gradi\"ent-afdalingsmethoden door het "afstervende ReLU" probleem, convergeert het nieuw algoritme vaker. ADAM convergeert 14 keer op 60 trainingen, terwijl het nieuw algoritme 47 op 60 keer convergeert. Daar tegenover staat dat de looptijd en het geheugengebruik kwadratisch schalen met het aantal datapunten. Voor een voorspellingsprobleem met een groter aantal datapunten, de Santa Fe laser dataset, convergeert het nieuwe algoritme ook. De training duurt echter gemiddeld 20min, tegenover 20s voor ADAM.




% In future research different algorithmic frameworks can be explored instead of the Augmented Lagrangian method. ALso expanding the algorithm to handle different loss functions would make it more widely applicable, and implementing a batch training mode would improve scalability

% Een lijst van figuren en tabellen is optioneel
%\listoffigures
%\listoftables
% Bij een beperkt aantal figuren en tabellen gebruik je liever het volgende:
\listoffiguresandtables
% De lijst van symbolen is eveneens optioneel.
% Deze lijst moet wel manueel aangemaakt worden, bv. als volgt:
\chapter{List of Abbreviations and Symbols}
\section*{Abbreviations}
\begin{flushleft}
  \renewcommand{\arraystretch}{1.1}
  \begin{tabularx}{\textwidth}{@{}p{12mm}X@{}}
  	AD & Algorithmic Differentiation \\
  	ADAM & ADaptive Moment Estimation \\
  	AL & Augmented Lagrangian \\
  	ALM & Augmented Lagrangian Method \\
  	ANN & Artificial Neural Network \\
  	BP & Backpropagation \\
  	DNN & Deep Neural Network \\
  	GD & Gradient Descent \\
  	LS & Least Squares \\
  	MS & Multiple Shooting \\
  	DMS & Direct Multiple Shooting \\
  	MSE & Mean Squared Error \\
  	NLP & Non-Linear Program \\
  	OCP & Optimal Control Problem \\
  	SGD & Stochastic Gradient Descent \\ 
  \end{tabularx}
\end{flushleft}
\section*{Symbols}
\begin{flushleft}
  \renewcommand{\arraystretch}{1.1}
  \begin{tabularx}{\textwidth}{@{}p{12mm}X@{}}
  	$\beta$ & penalty parameter \\
  	$C(\cdot)$ & Cost function \\
 	$d_i$ & layer dimension \\
  	$\eta$ & tolerance parameter \\
 	$h(\cdot)$ & constraint function \\
 	$F(\cdot)$ & objective function \\
    $f_W(\cdot)$ & neural network function \\
    $J$ & Jacobian Matrix \\
    $\mathcal{L}_{\beta}$   & $\beta$-Augmented Lagrangian \\
    $\mathcal{N}(\cdot,\cdot)$ & Normal distribution \\
    $\sigma(\cdot)$ & neuron activation function \\
    $W_i$ & Weight matrix of layer $i$ \\
    $x$ & network input \\
    $y$ & network target \\
    $z$ & network state \\
  \end{tabularx}
\end{flushleft}

% Nu begint de eigenlijke tekst
\mainmatter

\chapter{Introduction}
\section{Artificial Neural Networks}

Artifical Neural Networks (ANNs) are a very popular machine learning model. They are known to be very expressive, leading to low statistical bias. With enough neurons, ANNs can approximate any function.  They are especially useful for learning from very large data sets. But it is not entirely clear what the optimization of an ANN converges to, as the loss surface is highly non-convex. Nonetheless a number of results show that for wide enough networks, there are few "bad" local minima.

ANNs are composed of 'neurons', which are in some ways analogous to biological neurons.Each neuron is a nonlinear function transforming the weighted sum of its inputs and a bias:
\begin{equation}
      y = \sigma(w_1x_1+w_2x_2+...+w_nx_n + b)
\end{equation}

$w_i$ are the weights, $x_i$ are the inputs to the neuron, which come either from a previous neuron, or are fed into the network, $\sigma$ is the activation function, and finally a bias $b$ is also added to the sum. This is the McCulloch-Pitts neuron model.[REF] The most commonly used activation function $\sigma$ is the Rectified Linear Unit (ReLU):

\begin{equation}
      \sigma(x) = x^+ = \max(0,x)
\end{equation}

Many other activation functions are possible, such as the sigmoid function ($\frac{1}{1+e{-x}}$) or the tansig function ($tanh(x)$).

A visual representation is shown in figure \ref{neural}b. A full network is built by connecting layers of neurons as shown in figure \ref{neural}a. An ANN can also be expressed as a combination of function composition and matrix multiplication, ignoring for a moment the bias vectors.

\begin{equation}
         f(W,x) = W_L\sigma(W_{L-1}\sigma(...W_1\sigma(W_0x)...))
\end{equation}

where $W_n$ are the matrixes of the connection weights and $L$ is the depth of the network. 





   \begin{figure}[b]
	\centering
	\includegraphics[width=0.49\textwidth]{network}
	%\caption{Feedforward Deep Neural Network. (Retrieved from https://towardsdatascience.com)}
	\includegraphics[width=0.49\textwidth]{neuron}
	\caption{Feedforward Deep Neural Network and Single Neuron - McCulloch-Pitts model. (Retrieved from https://towardsdatascience.com)}
	\label{neural}
	\end{figure}

\newpage

\section{Neural Network Training}
Training a neural network is an optimization problem as we will discuss in this section. Ruoyu Sun covers in \cite{sun2019optimization} the current theory and algorithms for optimizing deep neural networks, upon which much of this section is based.

In a supervised learning problem a dataset of inputs and desired outputs is given: $x_i \in \mathbb{R}^{d_x}, y_i \in \mathbb{R}^{d_y}, i = 1,\dots,n$ with $x_i$ the input vectors, $y_i$ the desired output vectors and $n$ the number of data points. We want the network to predict the output $y_i$ based on the information in $x_i$, i.e. we want the network to learn the underlying mapping that connects the data. A standard fully connected network can be expressed as a combination of function composition and matrix multiplication as follows:


\begin{equation}
         f_W(x) = W_L\sigma(W_{L-1}\sigma(...W_1\sigma(W_0x)...))
\end{equation}

where $L$ is the depth of the network, $W_j$ are matrixes of dimension $d_j \times d_{j-1}, j=1 \dots L$ containing the connection weights and $\sigma$ is the activation function. The bias vectors $b_i$ have been omitted from this equation for clarity.

We want to pick the parameters of the neural network so that the predicted output $\hat{y}_i = f_W(x_i)$ is as close as possible to the true output $y_i$ for a certain distance metric $\mathit{l(\cdot,\cdot)}$. Thus the optimization problem can be written as follows:

\begin{equation}
\begin{aligned}
& \underset{W}{\text{minimize}}
& F(W) &= \sum\limits_{j=0}^{N}l(y_j,f_W(x_j)) \\
\end{aligned}
\label{op-eq}
\end{equation}

In this thesis only regression problems will be considered, where $l(x,y)$ is the quadratic loss function $l(x,y) = ||x^2-y^2||$. For classification problems cross entropy loss is the most common cost function: $l(x,y) = x\log(y)+(1-x)\log(y)$.

Most methods for solving equation \ref{op-eq} are based on gradient descent (GD). This algorithm uses the gradient of the loss function to search for a local minimum:
\begin{equation}
W_{k+1} = W_{k} - \eta_k\nabla F(W_k)
\label{gd-eq}
\end{equation}

where $\eta_k$ is the step size (a.k.a. "learning rate") and $\nabla F(W_k)$ is the gradient of the loss function at the $k$-th iterate.  

\section{Backpropagation}
Backpropagation is the current standard algorithm for calculating the gradient in equation \ref{gd-eq}. It was discovered and popularised in the context of neural networks by Rumelhart, Hinton \& Williams (1986) \ref{Rumelhart1986}, but in the context of optimal control the techniques for calculating this gradient were discovered in 1960 by Kelley \ref{} and Bryson \ref{}. It has been shown that reformulatin


Training of a network is done by minimizing the loss function $C$. For regression problems this will be often be a squared error loss, while for classification tasks the cross entropy is used.

\begin{equation*}
\begin{aligned}
& \underset{W}{\text{minimize}}
& C(W) &= \sum\limits_{j=0}^{N}||f(W,x^j) - y^j||^2 \\
\end{aligned}
\end{equation*}

with $x^j$ the input vectors, $y^j$ the target output and $N$ the number of data points

Backpropagation computes the gradient of the loss function $C$ with respect to the weight matrices $W$. It has two steps: in the first step the output of the network is calculated using the current weights and inputs. The activation values and derivatives of each neuron are stored. Then the error and gradient are calculated from the end of the network to the start, i.e. backwards.

Finally any gradient descent method can be used to find a step update. This is repeated until a stopping criterion is met.


\section{Neural Networks in Optimal Control Theory}

\begin{table}
\begin{tabular}{c | c | c }
Optimal Control & Neural Network & Notation\\ \hline
decision variables & weight parameters & $W$\\
state variables & (neuron) activation & $z$\\
\end{tabular}
\end{table}

A neural network can also be interpreted as a dynamical system.
\begin{equation*}
	\begin{aligned}
	z_0 &= x \\
	z_{k+1} &= \sigma(W_kz_k), & k = 0,...,L-1 \\
	y &= W_Hz_L \\
	\end{aligned}
\end{equation*}

In this way optimization methods from control theory can be applied. In particular, training a neural network can be formulated as the following Optimal Control Problem (OCP)

\begin{equation}
	\begin{aligned}
	& \underset{W}{\text{minimize}}
	& & \sum\limits_{j=0}^{N}||W_Lz_L^j - y^j||^2 \\
	& \text{subject to}
	& & z_{k+1}^j = \max(W_kz_k^j + b^j,0), &k = 0,\ldots,L-1,j = 1,\ldots,N
	\end{aligned}
	\label{ocp-eq}
\end{equation}


\section{Simultaneous Approach}
There are two main direct approaches to solving an OCP. First is the sequential approach, where the states are eliminated using the dynamics. This is equivalent to the backpropagation algorithm which is the current standard method. [ref mizutani]

The other approach is the simultaneous approach, where the state variables and the dynamics are kept as constraints. In control theory this approach often works better for highly nonlinear problems, which is certainly the case for training neural networks. The simultaneous approach is novel to neural networks and will be the topic of this thesis.

The disadvantage of this method is the number of variables that need to be optimized is much larger. For a fully connected neural network of width $W$, each layer will contain $W^2$ weights. Combined with a depth $D$, that gives approximately $W^2D$ weight variables to be optimized for both the backpropagation and simultaneous approach. Adding the states as variables however adds another $WDN$ variables, where $N$ is the number of samples in a training batch. The advantage of this method is that relaxing the states makes the problem more smooth, and will hopefully allow the optimization to converge more often to a good solution and not land in a bad local minimum.

\section{Goal of the Thesis}

\begin{itemize}
\item implement simultaneous approach

\item compare speed

\item compare reliability/convergence
\end{itemize}




\chapter{Initial exploration}
\label{cha:1}
In this chapter an initial exploration of the problem is done. This was done in MATLAB with the neural network toolbox for the backpropagation algorithm, and YALMIP with fmincon for the simultaneous approach.

\section{First experiment}
The first experiment conducted was to apply both algorithms to a test problem. A small neural network is constructed with 2 hidden layers with each layer containing 3 nodes with tansig activation function. This network is then trained to approximate a piece of a sine function. 

The backpropagation algorithm converges every time for this problem, taking anywhere between 10 and 1000 iterations. The stopping criteria is that the validation error increases for 6 consecutive iterations.

The simultaneous approach also converges if the intial conditions are set correctly. The weight variables are randomly initialized and the state variables are initialized by simulating the network once using the input vector. Letting fmincon run for 200 iterations gives a good solution every time, but the algorithm runs much slower. Its not known if this is due to difference in optimization and the fact that the backpropagation algorithm is running on my GPU instead of the CPU.

\section{Second experiment}
For this experiment the same test problem is considered but with a different network. Here a shallow network 1 layer deep and 10 neurons wide is considered, with the ReLU activation function.

The backpropagation function will sometimes get stuck in a bad solution while training this network, but usually finds a good approximation.

For the simultaneous approach the constraints will first have to be adapted a bit for the ReLU function. The ReLU function can be transformed into smooth constraints as follows:

   \begin{gather*}
   x_{k+1}^j = \max(W_kx_k^j,0) \\
   \Updownarrow \\
   x_{k+1}^j = -\min(-W_kx_k^j,0) \\
   \Updownarrow \\
   \min(x_{k+1}^j-W_kx_k^j) = 0 \\
   \Updownarrow \\
   (x_{k+1}^j-W_kx_k^j)^\top x_{k+1}^j = 0,\\
   x_{k+1}^j\geq 0,x_{k+1}^j-W_kx_k^j\geq 0
   \end{gather*}
   
Even using these smooth constraints, the algorithm very often gets stuck in a bad local minimum.
\section{Further exploration}
There are many options that can be explored to further compare these algorithms.

\begin{itemize}
\item Activation function: ReLU is the most popular activation function in the field. Others are tansig, sigmoid, SoftPlus, leaky ReLU, etc.
\item Network size: Networks can be up to thousands of neurons wide and hundreds of layers deep. 
\item Network architecture: There is a huge variety of possible network architectures. Convolutional neural networks for example are very popular for image recognition.
\item Test problems: Neural networks have many applications. Some applications could benefit more from this training method than others.
\item Optimization algorithm: fmincon is quite a general method, a more specific method might perform better
\item Stopping criteria and initial conditions

\end{itemize}




%%% Local Variables: 
%%% mode: latex
%%% TeX-master: "thesis"
%%% End: 

\chapter{Augmented Lagrangian Method}
\label{cha:2}
In this chapter the direct multiple shooting approach is examined more closely and a more specific algorithm is designed to replace \texttt{fmincon}, which is a very general method. For this problem the Augmented Lagrangian Method has been chosen. This is a common method for solving constrained Non Linear Programs (NLPs). Instead of using the classical method, an Augmented Lagrangian framework is adapted from a recent paper\cite{sahin2019}.It will be implemented in python using \texttt{numpy} v1.20.2, \texttt{scipy} v1.6.3  and \texttt{tensorflow} v2.5.0 \cite{numpy},\cite{scipy},\cite{tensorflow}.

\section{Classical Augmented Lagrangian Method}
The Augmented Lagrangian Method (ALM) is a classical algorithmic framework for solving constrained NLPs. It was first discovered in 1969 \cite{Hestenes1969},\cite{Powell1969} and was known as the method of multipliers. Textbook examples of this method can be found in \cite{Birgin2009} and \cite{bertsekas2014constrained}.

It is designed to minimize equality constrained optimization problems defined in the following way:

\begin{equation}
	\begin{aligned}
	& \underset{u}{\text{min}} & f(u) & \\
	& \text{s.t.} & h(u) &= 0 \\
	\end{aligned}
\end{equation}

ALM solves this by minimizing a series of unconstrained problems in a similar manner as the penalty method. In each iteration a $\beta$-augmented Lagrangian $\mathcal{L}_\beta(x,\lambda)$ is minimized for x:

\begin{equation}
	\underset{u}{\text{min}} \hspace{.5em} \underset{\lambda}{\text{max}} \hspace{.5em}  \mathcal{L}_\beta(u,\lambda) = f(u) + \langle\lambda,h(u)\rangle + \frac{\beta}{2} || h(u) ||^2_2
\end{equation}

where $\beta>0$ is the penalty weight. This can be viewed as a penalty method which has been shifted using the term in $\lambda$\cite{Birgin2009}. When $\beta$ or $\lambda$ tend to infinity, $h(u)$ will be forced to zero, leading the Lagrangian to converge to the same solution as the original problem.

The algorithm proceeds as follows:
\begin{gather*}
	u_{k+1} = \underset{u}{\text{argmin}} \hspace{.5em} \mathcal{L}_\beta(u,\lambda_k) \\
	\lambda_{k+1} = \lambda_k + \sigma_k h(u_{k+1})
\end{gather*}

where $\sigma_k$ is the step size at iteration $k$. Then in each step the penalty parameter $\beta_k$ is increased or kept the same, depending on the size of the constraint violation. This continues until an acceptable solution has been found:

\begin{equation}
	||h(u_k)|| \leq \tau_1 \hspace{.5em}\text{and}\hspace{.5em} ||\nabla_u\mathcal{L}_{\beta_k}(u_k,\lambda_k)|| \leq \tau_2
\end{equation}

with $\tau_1,\tau_2$ the chosen tolerances.

\section{Applied Augmented Lagrangian Method}

The OCP equation \ref{ocp-eq} of training a neural net with MSE loss function is a constrained nonlinear least squares(LS) problem:
\begin{equation*}
	\begin{aligned}
	& \underset{W}{\text{minimize}}
	& & \sum\limits_{j=0}^{n}||\sigma_L(W_Lz_L) - y_j||^2_2 \\
	& \text{subject to}
	& & z_{1,j} = \sigma_0(W_0,x_j), &j = 1,\ldots,n \\
	& & & z_{k+1,j} = \sigma_k(W_kz_{k,j}), &k = 1,\ldots,L-1,j = 1,\ldots,n \\
    & & \Updownarrow \\
	& \text{min}
	&  & \frac{1}{2} ||F(u)||^2_2 \\
	& \text{s. t.}
	& &  h(u) = 0
	\end{aligned}
\end{equation*}
Where $u = \{W,z\}$ is the collection of both the weight and state variables into a single vector.

The subproblem that will be solved in each iteration is then:
\begin{equation}
	\begin{aligned}
	 & \underset{u}{\text{argmin}} & \mathcal{L}_{\beta}(u,\lambda) 
	     &= \frac{1}{2} ||F(u)||^2_2 + \langle\lambda,h(u)\rangle + \frac{\beta}{2} || h(u) ||^2_2 \\
	 & & &= \frac{1}{2} ||F(u)||^2_2 + \frac{\beta}{2} ||h(u) + \lambda/\beta ||^2_2 - \frac{1}{2\beta} ||\lambda||^2_2 \\
	 & & &= \frac{\beta}{2} \Big|\Big|
		\begin{bmatrix}
			F(u)/\sqrt{\beta} \\
			h(u) + \lambda/\beta
		\end{bmatrix} \Big|\Big|^2_2 \\
	\end{aligned}
	\label{loss}
\end{equation}

Instead of using the textbook algorithm, an algorithmic framework from a more recent paper\cite{sahin2019} is adapted to the problem, shown in Algorithm \ref{algo}.

\begin{algorithm}[H]
\SetAlgoLined
\SetKw{Kw}{Initialization}
\SetKwComment{Comment}{}{}
\SetCommentSty{emph}
\KwIn{Initial weights vector $W$, penalty parameter $\beta$, stopping tolerance $\tau$, input-target pairs $(x_i,y_i), i = 1,\ldots,n$}
\Kw{$u_0 = \{W,f_{W}(x)\}, \lambda_0 \in \mathcal{N}(0,1)$}
\Comment*[r]{Initialize state variables by simulating network, initialize dual variables randomly}
\For{k = 0,1,...}{
 	$\eta_k = 1/\beta^k$
 	\Comment*[r]{Update tolerance}
 	find $u_{k+1}$ such that \\
 	\Indp$||\nabla_{u_k}\mathcal{L}_{\beta^k}(u_k,\lambda_k)|| \leq \eta_k$ \label{ls-prob}
 	\Comment*[r]{Approx. primal solution}
 	\Indm$\sigma_{k+1} = \text{min}\big(\frac{||h(u_0)||\log^22}{||h(u_{k+1})||k \log^2(k+1)},1\big)$
 	\Comment*[r]{Update dual step size}
 	$\lambda_{k+1} = \lambda_k + \sigma_{k+1}h(u_{k+1})$
 	\Comment*[r]{Update dual variables}
 	$||\nabla_{u_{k+1}}\mathcal{L}_{\beta^k}(u_{k+1},\lambda_k)|| + ||h(u_{k+1})||<\tau$
 	\Comment*[r]{Stopping Criterion}
 	
 }
 \caption{Inexact Augmented Lagrangian Method}
 \label{algo}
\end{algorithm}
The penalty parameter increases geometrically, $\beta_k = \beta_0^k$, and the tolerence decreases geometrically $\eta_k = 1/\beta_k$. It is called the inexact Augmented Lagrangian Method (iALM) because the optimizer $u^*$ of subproblem \ref{loss} can only be solved to an approximate solution.  The choice of dual step size $\sigma_k$ is to ensure the boundedness of the dual variables $\lambda_k$ \cite{sahin2019},\cite{bertsekas1976}. 


Figure \ref{nabla} shows the convergence behaviour of this algorithm. In this figure the algorithm was run for 10 epochs on a neural network training problem. The gradient of the $\beta$-Augmented Lagrangian is plotted in blue.

\begin{equation}
||\nabla_{u_k}\mathcal{L}_{\beta^k}(u_k,\lambda_k)|| = ||2(\nabla_{u_k}\begin{bmatrix} F(u)/\sqrt{\beta} \\ h(u) + \lambda/\beta \end{bmatrix})\mathcal{L}_{\beta^k}(u_k,\lambda_k)||
\end{equation}

The gradient decreases geometrically as the tolerence is decreased in each step $\eta_{k+1} = \eta_k/\beta$. The MSE loss of the network is plotted in red. It is calculated by taking the current optimal weights at that epoch and simulating the network on the training data. In this example the MSE loss reaches a minimum after 6 iterations, which is a typical result.

The constraint violations, and the variables associated with the states are not relevant when evaluating the performance of the network. For this reason the stopping criterion in Algorithm \ref{algo} may not be the most practical choice. In deep learning many different stopping criteria are used. Often these are based on the training loss, or the loss on a validation set which is held apart from the training data. Usually a tradeoff will have to be made between training performance and overfitting the data (Goodfellow et al. \cite{Goodfellow-et-al-2016}, Sec. 8.1). This is a practical issue, in the next chapter the problem of choosing an appropriate stopping criterion is examined more fully.

\begin{figure}
	\centering
	\includegraphics[width=.7\textwidth]{nabla}
	\caption{Typical convergence behaviour of Algorithm \ref{algo}}
	\label{nabla}
\end{figure}

\section{Least Squares Solver}
To solve the LS problem \ref{loss} in Algorithm \ref{algo}, a Trust Region Reflective(\texttt{trf}) method is used, which is implemented in \texttt{scipy.optimize.least\_squares} \cite{scipyls}. The following description is given by \texttt{scipy}: "The algorithm iteratively solves trust-region subproblems augmented by a special diagonal quadratic term and with trust-region shape determined by the distance from the bounds and the direction of the gradient." The \texttt{trf} method is described as being robust for both bounded and unbounded problems, and well suited for sparse Jacobians. 

The LS problem \ref{loss} is an unbounded problem for which the library recommends using a Levenberg-Marquardt method. However this method cannot handle cases where the Jacobian has more columns than rows, which can sometimes occur depending on the size of the network and the number of data points used for training. Therefore we cannot use the Levenberg-Marquardt algorithm.

To efficiently solve the least squares problem, the solver requires an analytical solution for the Jacobian, which will be explained in the next section.

\section{Jacobian}

To solve the least squares problem, the Jacobian matrix of $M_{\beta}(u,\lambda) = \begin{bmatrix} F(u)/\sqrt{\beta} \\ h(u) + \lambda/\beta \end{bmatrix}$ must be calculated. A Jacobian is the matrix of all partial derivatives of a vector valued function. In this case:

\begin{equation}
J_{M_{\beta}} = 
\begin{bmatrix}
\frac{\partial{M_{\beta}}}{\partial u_1} & 
\frac{\partial{M_{\beta}}}{\partial u_2} & ... & 
\frac{\partial{M_{\beta}}}{\partial u_n} \\
\end{bmatrix}
\end{equation}

 It has a relatively sparse structure because there are no distant connections in the neural net, each layer is only connected to the next one and the previous one. In this section the partial derivative associated with each variable will be presented.
 
 First the columns of $J_{M_{\beta}}$ associated with the weight variables will be examined:
 
\begin{equation}
	\frac{\partial{M_{\beta}}}{\partial W_k} = -z_{k,j}\sigma'_k(W_kz_{k,j}), k = 0,\ldots,L, j = 1,\ldots,n
\end{equation}

$W_k$ is a matrix of size $d_{k+1}\times(d_k+1)$, and $z_{k,j}$ are vectors of size $d_k+1$, therefore this evaluates as a 3D tensor. $W_k$ must be vectorized first to allow this derivative to be used in the Jacobian. After vectorization the dimensions of this partial derivative as a block matrix are: $d_{k+1}n \times d_{k+1}(d_{k}+1)$. The last partial derivative $\frac{\partial{M_{\beta}}}{\partial W_L}$ is also multiplied by a factor $\frac{1}{\beta}$.

 Next the columns of $J_{M_{\beta}}$ associated with the states $z_k$ are examined:
 
 \begin{equation}
 	\frac{\partial{M_{\beta}}}{\partial z_k} = \begin{bmatrix} 1 \\ -W_k\sigma'_k(W_kz_k) \end{bmatrix}, k = 1,\ldots,L
 \end{equation}
 $W_k$ are as before and $z_k$ are matrices of size $(d_k+1)\times n$. The row of ones associated with the biases vector in $W_k$ is not a variable so it is excluded from the jacobian. After vectorization of $z_k$ the partial derivative has dimensions $(d_k+d_{k+1})n\times d_kn$.
 
For a fully connected neural network with identity output activation an example has been written out in Table \ref{jac-tab}. Figure \ref{jac} shows a visual representation of the matrix, where the nonzero elements have been colored black. An alternative representation, which is mathematically the same is to swap the rows corresponding to the loss function $F(u)$ to the bottom. This gives a matrix with a banded structure, which is is plotted in figure \ref{jac2}. This is the representation used in the code.

\begin{figure}[p]
	\centering
	\begin{subfigure}{\textwidth}
	  \centering
	  \includegraphics[width=\textwidth]{jac0.png}
	  \caption{Jacobian matrix}
	  \label{jac}
	\end{subfigure}
	\begin{subfigure}{\textwidth}
	  \centering
	  \includegraphics[width=\textwidth]{jac1.png}
	  \caption{Jacobian matrix, rearranged}
	  \label{jac2}
	\end{subfigure}
	\caption{Visual representation of Jacobian matrix for network with 2 inputs, 2 outputs, width 3, depth 2 and 7 datapoints. The non-zero elements have been colored black. The Jacobian at the top is the same as the one on the bottom, but with swapped rows.}
	\label{jactot}
\end{figure}

Consider a feedforward network with input dimension I, an output dimension O, and D hidden layers of width W. The weight matrixes have $I \cdot W + O \cdot W + (D-1) \cdot W \cdot W$ parameters, the bias vectors have $D \cdot W+O$ parameters and the state vectors have $D \cdot W \cdot N$ parameters. On the other hand $M_{\beta}(u,\lambda)$ has an output dimension of $D \cdot W \cdot N + O \cdot N$. The dimension of the Jacobian for this network is therefore :
\begin{equation}
(D \cdot W \cdot N + O \cdot N) \times (D \cdot W \cdot N + O + (D+I+O) \cdot W + (D-1) \cdot W^2)
\label{jdim}
\end{equation}. 

The Jacobian scales quadratically in size with the depth of the network and the number of datapoints. It scales cubically with the width of the network. The Jacobian will have more rows than columns when:
\begin{equation}
	N \geq 1 + (D+I+O) \cdot W/O + (D-1) \cdot W^2/O
\end{equation}



\section{Numerical verification of Jacobian Matrix}
In the previous section the Jacobian matrix was derived analytically. In this section will be explained how the Jacobian is verified algorithmically.

Algorithmic Differentation (AD) is a set of techniques which can be used to calculate the derivative of any computer code \cite{Rall1981}, \cite{wikiad}. Because all code is composed of elementary operations, AD can use the chain rule alongside the operations to automatically compute derivatives of arbitrary order. By injecting code from an AD library into the calculation of the neural network, the Jacobian can be calculated numerically. For this the AlgoPy python library was used \cite{algopy}. The output of the AD was then compared to the analytical result for a number of different network configurations, confirming them to be equal within a small tolerance. The code is shown in Appendix \ref{AD}.

\begin{table}[p]
\tiny
\centering

\begin{subtable}{\textwidth}
\makebox[\textwidth][c]{
\begin{tabular}{r r | c c c c c}

\multicolumn{7}{c}{Weight variables, each entry is a block matrix} \\ \hline

& $\nabla^T_{W_0,b_0}M$ & $W_{0_1}$ & $W_{0_2}$ & ... & $W_{0_W}$ & $b_0$ \\
& dim & I & I &...& I & W \\ \hline
$F$ & O*N & 0 & 0 &...& 0 & 0\\ \hline
$h_1$ & N & 		$-x\sigma'(W_{0_1}x+b_{0_1})$ & 0 &...& 0 & $-\sigma'(W_{0_1}x+b_{0_1})$ \\
      & N & 0 & 	$-x\sigma'(W_{0_2}x+b_{0_2})$ &...& 0 &  	$-\sigma'(W_{0_2}x+b_{0_2})$ \\
      &...&...&...&...&...&... \\
      & N & 0 & 0 &...& $-x\sigma'(W_{0_W}x+b_{0_W})$ &  		$-\sigma'(W_{0_W}x+b_{0_W})$ \\ \hline
$h_2$ & W*N & 0 & 0 &...& 0 & 0 \\
...   & ... &...&...&...&...&...\\ 
$h_{D}$ & W*N & 0 & 0 &...& 0 & 0 \\ \hline \\ \hline

& $\nabla^T_{W_i,b_i}M$ & $W_{i_1}$ & $W_{i_2}$ &...& $W_{i_W}$ & $b_i$ \\
& dim & W & W &...& W & W \\ \hline
$F$ & O*N & 0 & 0 &...& 0 & 0 \\ \hline
$h_1$ & W*N & 0 & 0 &...& 0 & 0 \\
...   & ... &...&...&...&...&...\\ \hline
$h_{i+1}$ & N & 		$-z_1\sigma'(W_{i_1}z + b_{i_1})$ & 0 &...& 0 & $-\sigma'(W_{i_1}x+b_{i_1})$ \\
      & N & 0 & 	$-z_1\sigma'(W_{i_2}z + b_{i_2})$ &...& 0 & 	$-\sigma'(W_{i_2}x+b_{i_2})$ \\
      &...&...&...&...&...&... \\
      & N & 0 & 0 &...& $-z_1\sigma'(W_{i_W}z + b_{i_W})$ & 		$-\sigma'(W_{i_W}x+b_{i_W})$ \\ \hline
...   & ... &...&...&...&...&...\\ 
$h_{D}$ & W*N & 0 & 0 &...& 0 & 0 \\ \hline \\ \hline

& $\nabla^T_{W_D,b_D}M$ & $W_{D_1}$ &  $W_{D_2}$  &...&  $W_{D_O}$ & $b_D$ \\
& dim & W & W &...& W & O \\ \hline
$F$ & N & $-\frac{z_D}{\sqrt{c}}\sigma_O'(W_{D_1}x+b_{D_1})$ & 0 &...& 0 & $-\frac{1}{\sqrt{c}}\sigma_O'(W_{D_1}x+b_{D_1})$ \\
    & N & 0 & $-\frac{z_D}{\sqrt{c}}\sigma_O'(W_{D_2}x+b_{D_2})$ &...& 0 & $-\frac{1}{\sqrt{c}}\sigma_O'(W_{D_2}x+b_{D_2})$ \\
      &...&...&...&...&...&... \\
    & N & 0 & 0 &...& $-\frac{z_D}{\sqrt{c}}\sigma_O'(W_{D_O}x+b_{D_O})$ & $-\frac{1}{\sqrt{c}}\sigma_O'(W_{D_O}x+b_{D_O})$ \\ \hline
$h_1$ & W*N & 0 & 0 &...& 0 & 0 \\
...   & ... &...&...&...&...&...\\ 
$h_{D}$ & W*N & 0 & 0 &...& 0 & 0 \\ \hline
      
\end{tabular}}
\end{subtable}


\begin{subtable}{\textwidth}
\makebox[\textwidth][c]{
\begin{tabular}{ r r | c c c c }
\multicolumn{6}{c}{State variables, each entry is a diagonal matrix} \\ \hline
& $\nabla^T_{z_i}M$ & $z_{i_1}$ & $z_{i_2}$ &...& $z_{i_W}$\\
&  dim & N & N &...& N \\ \hline
$F$ & O*N & 0 & 0 &...& 0 \\ \hline
$h_1$ & W*N & 0 & 0 &...& 0 \\
...   & ... &...&...&...&...\\\hline
$h_i$ & N & 1 & 0 &...& 0 \\
      & N & 0 & 1 &...& 0  \\
      &...&...&...&...&...\\ 
      & N & 0 & 0 &...& 1  \\ \hline
$h_{i+1}$ & N & $-W_{i_{1,1}}\sigma'(W_{i_1}z_i+b_{i_1})$ & $-W_{i_{1,2}}\sigma'(W_{i_1}z_i+b_{i_1})$ &...& $-W_{i_{1,W}}\sigma'(W_{i_1}z_i+b_{i_1})$\\
          & N & $-W_{i_{2,1}}\sigma'(W_{i_2}z_i+b_{i_2})$ & $-W_{i_{2,2}}\sigma'(W_{i_2}z_i+b_{i_2})$ &...& $-W_{i_{2,W}}\sigma'(W_{i_2}z_i+b_{i_2})$\\
      &...&...&...&...&...\\ 
          & N & $-W_{i_{W,1}}\sigma'(W_{i_W}z_i+b_{i_W})$ & $-W_{i_{W,2}}\sigma'(W_{i_W}z_i+b_{i_W})$ &...& $-W_{i_{W,W}}\sigma'(W_{i_W}z_i+b_{i_W})$\\ \hline
...   & ... &...&...&...&...\\ 
$h_{D}$ & W*N & 0 & 0 &...& 0 \\ \hline \\
& $\nabla^T_{z_D}M$ & $z_{D_1}$ & $z_{D_2}$ &...& $z_{D_W}$\\
& dim & N & N & ... &  N \\ \hline
F & N &         $-W_{D_{1,1}}\sigma_O'(W_{D_1}z_D+b_{D_1})$ & $-W_{D_{1,2}}\sigma_O'(W_{D_1}z_D+b_{D_1})$ &...& $-W_{D_{1,W}}\sigma_O'(W_{D_1}z_D+b_{D_1})$\\
          & N & $-W_{D_{2,1}}\sigma_O'(W_{D_2}z_D+b_{D_2})$ & $-W_{D_{2,2}}\sigma_O'(W_{D_2}z_D+b_{D_2})$ &...& $-W_{D_{2,W}}\sigma_O'(W_{D_2}z_D+b_{D_2})$\\
      &...&...&...&...&...\\ 
          & N & $-W_{D_{O,1}}\sigma_O'(W_{D_O}z_D+b_{D_O})$ & $-W_{D_{O,2}}\sigma_O'(W_{D_O}z_D+b_{D_O})$ &...& $-W_{D_{O,W}}\sigma_O'(W_{D_O}z_D+b_{D_O})$\\ \hline
$h_1$ & W*N & 0 & 0 &...& 0 \\
...   & ... &...&...&...&...\\\hline
$h_D$ & N & 1 & 0 &...& 0 \\
      & N & 0 & 1 &...& 0  \\
      &...&...&...&...&...\\ 
      & N & 0 & 0 &...& 1  \\ \hline
\end{tabular}}
\end{subtable}
\caption{Jacobian of feedforward neural network. In this table the biases are not included in the weight matrices $W_k$. Each layer has the same width $W$ and same activation $\sigma$. There are $D$ layers. The input dimension is $I$ and the output dimension is $O$. There are $N$ datapoints.}
\label{jac-tab}

\end{table}
\section{Conclusion}
In this chapter an Augmented Lagrangian framework was proposed to solve the Optimal Control Problem. A Least Squares solver was applied to the LS problem at the heart of the algorithm, and a Jacobian Matrix was analytically derived which is supplied to the solver. The Jacobian was also algorithmically verified. In the next chapter the algorithm will be investigated using numerical tests and compared against an industry standard optimizer.




\chapter{Tests}
\label{cha:3}
This chapter discusses several tests to compare the newly proposed method with backpropagation. Each test will be described in detail. The reference code to which we compare our experiments is the backpropagation method of Tenserflow.

The code was executed on ...


\section{The First Topic of this Chapter}
\subsection{Item 1}
\subsubsection{Sub-item 1}

\subsubsection{Sub-item 2}

\subsection{Item 2}

\section{The Second Topic}

\section{Conclusion}

%%% Local Variables: 
%%% mode: latex
%%% TeX-master: "thesis"
%%% End: 

% ... en zo verder tot
\include{chap-n}
\chapter{Conclusion}
\label{cha:conclusion}
The main goal of this thesis was to present a novel algorithm for training deep neural networks, using methods commonly used in Optimal Control Theory. The optimization problem of neural network training was reformulated into an Optimal Control Problem. In a first step the direct multiple shooting method was applied to the OCP, which is commonly used in control theory for very nonlinear problems. 

This method was implemented in MATLAB using a very general optimization function \texttt{fmincon}, proving that the problem was feasible. Then an Augmented Lagrangian Method was presented to solve the OCP. First a textbook ALM method was implemented, later this was improved by using an inexact ALM detailed in \cite{}. The Jacobian matrix at the center of the ALM was verified to be correct using automatic differentiation tools. This ALM method was then written into code using python with \texttt{numpy},\texttt{scipy}, and \texttt{keras}

Finally the novel algorithm was tested against industry standard backpropagation methods, ADAM and SGD. It compares favorably for some smaller, harder training problems. (SPECIFY) However it scales poorly for larger datasets. As the title indicates, the original goal of this new algorithm was to better avoid "bad" local minima in training. But the new algorithm does not show much improvement compared to standard methods, and practice has shown that is not a common problem. In larger neural networks, experts suspect local minima usually have comparable performance to the global minimum \cite{Goodfellow-et-al-2016}.  


TODO: Batch approach. It also cannot yet handle loss functions besides MSE, which could be the topic of further research.




%%% Local Variables: 
%%% mode: latex
%%% TeX-master: "thesis"
%%% End: 


% Indien er bijlagen zijn:
\appendixpage*          % indien gewenst
\appendix
\chapter{Source code}
\label{app:A}
This appendix contains the source of MATLAB code used in the thesis.  This code can also be found at \url{https://github.com/jan-scheers/thesis/}.

\section{First experiment source}
This script executes a batch test for the first MATLAB experiment. This experiment uses the \texttt{tansig} ($\tanh$) activation function.
\begin{verbatim}
R = zeros(20,6);
for k = 1:20
    
    N = 20;
    x = linspace(0,1,N);
    y = -sin(0.8*pi*x)+normrnd(0,0.1,size(x));
    [trainInd,valInd,testInd] = dividerand(N,0.8,0,0.2);

    
    net = fitnet([3 3]);
    net = configure(net,x,y);
    net.inputs{1}.processFcns = {};
    net.outputs{2}.processFcns = {};
    net.divideFcn = 'divideind';
    net.divideParam.trainInd = trainInd;
    net.divideParam.valInd = valInd;
    net.divideParam.testInd = testInd;
    
    
    N = 0.8*N;
    xin = x(trainInd);

    w1 = sdpvar(3,1);
    b1 = sdpvar(3,1);
    x1 = sdpvar(3,N);
    w2 = sdpvar(3,3,'full');
    b2 = sdpvar(3,1);
    x2 = sdpvar(3,N);
    w3 = sdpvar(1,3);
    b3 = sdpvar(1,1);


    assign(w1,net.IW{1});
    assign(b1,net.b{1});
    assign(x1,tansig(value(w1*xin+repmat(b1,1,N))));
    assign(w2,net.LW{2,1});
    assign(b2,net.b{2});
    assign(x2,tansig(value(w2*x1 +repmat(b2,1,N))));
    assign(w3,net.LW{3,2});
    assign(b3,net.b{3});

    res = w3*x2 + b3 - y(trainInd);
    obj = res*res';
    
    net.trainFcn = 'traingd';
    net.trainParam.epochs = 2000;
    net.trainParam.max_fail = 200;
    [net, tr] = train(net,x,y);

    con = [x1 == tansig(w1*xin+repmat(b1,1,N));
           x2 == tansig(w2*x1 +repmat(b2,1,N))];
    ops = sdpsettings('usex0',1,'solver','fmincon');
    ops.fmincon.MaxFunEvals = 20000;
    ops.fmincon.MaxIter = 40;

    t0 = cputime;
    optimize(con,obj,ops);
    t1 = cputime-t0;
    
    x1s = tansig(value(w1)*x(testInd)+value(b1));
    x2s = tansig(value(w2)*x1s+value(b2));
    ysm = value(w3)*x2s+value(b3);
    
    R(k,:) = [tr.perf(end),tr.tperf(end),tr.time(end),value(obj),immse(ysm,y(testInd)),t1]
end
%%
% figure(2);
% hold off
% 
% plot(x,-sin(.8*pi*x),'b-','Linewidth',2)
% hold on;
% 
% xsm = linspace(0,1,1000);
% x1s = tansig(value(w1)*xsm+value(b1));
% x2s = tansig(value(w2)*x1s+value(b2));
% ysm = value(w3)*x2s+value(b3);
% 
% plot(xsm,ysm,'r-','Linewidth',2);
% 
% plot(x(trainInd),y(trainInd),'k+','Linewidth',2,'Markersize',8);
% plot(x(testInd) ,y(testInd) ,'r+','Linewidth',2,'Markersize',8);
% 
% ylim([-1.2,0])
% 
% legend("-sin(.8\pix)","Neural net fit","Training data","Test data")
\end{verbatim}

\subsection{Second experiment source}
This script executes a batch test for the second MATLAB experiment. This experiment uses the \texttt{poslin} (ReLU) activation function.
\begin{verbatim}
R = zeros(20,6);
for k = 1:20
    
    N = 20;
    x = linspace(0,1,N);
    y = -sin(0.8*pi*x)+normrnd(0,0.1,size(x));
    [trainInd,valInd,testInd] = dividerand(N,0.8,0,0.2);

    
    net = fitnet([8 8]);
    net.inputs{1}.processFcns = {};
    net.outputs{2}.processFcns = {};
    net.layers{1}.transferFcn = 'poslin';
    net.layers{2}.transferFcn = 'poslin';
    net.divideFcn = 'divideind';
    net.divideParam.trainInd = trainInd;
    net.divideParam.valInd = valInd;
    net.divideParam.testInd = testInd;
    net = configure(net,x,y);
    
    
    N = 0.8*N;
    xin = x(trainInd);

    w1 = sdpvar(8,1);
    b1 = sdpvar(8,1);
    x1 = sdpvar(8,N);
    w2 = sdpvar(8,8,'full');
    b2 = sdpvar(8,1);
    x2 = sdpvar(8,N);
    w3 = sdpvar(1,8);
    b3 = sdpvar(1,1);

    assign(w1,net.IW{1});
    assign(b1,net.b{1});
    assign(x1,poslin(value(w1*xin+repmat(b1,1,N))));
    assign(w2,net.LW{2,1});
    assign(b2,net.b{2});
    assign(x2,poslin(value(w2*x1 +repmat(b2,1,N))));
    assign(w3,net.LW{3,2});
    assign(b3,net.b{3});
    
    f1 = (x1-(w1*xin+repmat(b1,1,N)));
    f2 = (x2-(w2*x1 +repmat(b2,1,N)));
    con = [f1 >= 0; x1 >= 0; f1.*x1 <= 0;
          f2 >= 0; x2 >= 0; f2.*x2 <= 0];

    res = w3*x2 + b3 - y(trainInd);
    obj = res*res';
    
    net.trainFcn = 'traingd';
    net.trainParam.epochs = 2000;
    net.trainParam.max_fail = 200;
    [net, tr] = train(net,x,y);

    ops = sdpsettings('usex0',1,'solver','fmincon');
    ops.fmincon.MaxFunEvals = 20000;
    ops.fmincon.MaxIter = 40;

    t0 = cputime;
    optimize(con,obj,ops);
    t1 = cputime-t0;
    
    x1s = poslin(value(w1)*x(testInd)+value(b1));
    x2s = poslin(value(w2)*x1s+value(b2));
    ysm = value(w3)*x2s+value(b3);
    
    R(k,:) = [tr.perf(end),tr.tperf(end),tr.time(end),value(obj),immse(ysm,y(testInd)),t1]
end
%%
% figure(2);
% hold off
% 
% plot(x,-sin(.8*pi*x),'b-','Linewidth',2)
% hold on;
% 
% xsm = linspace(0,1,1000);
% x1s = tansig(value(w1)*xsm+value(b1));
% x2s = tansig(value(w2)*x1s+value(b2));
% ysm = value(w3)*x2s+value(b3);
% 
% plot(xsm,ysm,'r-','Linewidth',2);
% 
% plot(x(trainInd),y(trainInd),'k+','Linewidth',2,'Markersize',8);
% plot(x(testInd) ,y(testInd) ,'r+','Linewidth',2,'Markersize',8);
% 
% ylim([-1.2,0])
% 
% legend("-sin(.8\pix)","Neural net fit","Training data","Test data")



\end{verbatim}

\chapter{Source code}
\label{app:B}
This appendix contains the source code of the Augmented lagrangian method and the automatic verification of the Jacobian Matrix
\footnotesize
\section{Augmented lagrangian method}
\begin{verbatim}
#alm.py
import numpy as np
import tensorflow as tf
from tensorflow import keras
from scipy import sparse
import scipy.optimize as op
import scipy.linalg as la

class Dense_d(keras.layers.Dense):
    def __init__(self,activation_,**kwargs):
        super().__init__(**kwargs)
        self.activation_ = activation_

class ALMModel:
    def __init__(self,model,x,y):
        self.model = model
        t = self.model(x)
        self.x = x
        self.y = y
        self.batch_size = x.shape[0]
        self.nz = sum([self.batch_size*model.layers[i].units for i in range(len(model.layers)-1)])

    def read(self,u):
        layers = self.model.layers
        w = []
        p = 0
        for i in range(0,len(layers)):
            s = layers[i].weights[0].shape
            m,n = s[1],s[0]+1
            w.append(u[p:p+m*n].reshape(m,n))
            p = p+m*n

        z = [self.x.transpose()]
        for i in range(1,len(layers)):
            m,n = w[i].shape[1]-1,self.batch_size
            zi = u[p:p+m*n].reshape(m,n)
            z.append(zi)
            p = p+m*n
        y = np.r_[self.y.transpose()]
        z.append(y)

        return w,z
    
    def write(self):
        u = np.empty(0)
        for layer in self.model.layers:
            w,b = layer.weights
            m,n = w.shape[1],w.shape[0]+1
            w = np.c_[w.numpy().transpose(),b.numpy().reshape(m,1)]
            u = np.append(u,w)
        z = [self.x]
        for i,layer in enumerate(self.model.layers):
            z.append(layer(z[i]))

        z = z[1:-1]
        z = [zi.numpy().transpose() for zi in z] 
        u = np.append(u,np.concatenate(z))
        return u

    def set_weights(self,w):
        for i,layer in enumerate(self.model.layers):
            wi = w[i].transpose()
            layer.set_weights([wi[:-1,:],wi[-1:,].reshape((-1,))])

    def h(self,u):
        w,z = self.read(u)
        h = []
        for i,layer in enumerate(self.model.layers[:-1]):
            zi_e = np.r_[z[i],np.ones((1,self.batch_size))]
            hi = z[i+1] - np.array(layer.activation(w[i].dot(zi_e)))
            h.append(hi.reshape(-1))
        return np.concatenate(h)

    def L(self,u,beta,l):
        w,z = self.read(u)
        
        out_layer = self.model.layers[-1]
        z_e = np.r_[z[-2],np.ones((1,self.batch_size))]
        F = z[-1] - np.array(out_layer.activation(w[-1].dot(z_e)))
        F = F.reshape(-1)/np.sqrt(beta)

        h = self.h(u)+l/beta
        return np.concatenate([h,F])

        
    def JL(self,u,beta):
        layers = self.model.layers
        w,z = self.read(u)

        w_ = []
        z_ = []
        np.set_printoptions(linewidth=200)
        for i in range(len(layers)):
            # Add bias to weights
            zi_e = np.r_[z[i],np.ones((1,self.batch_size))]

            ai = layers[i].activation_(w[i].dot(zi_e))

            wi_ = -zi_e*ai[:,np.newaxis,:]
            wi_ = wi_*np.eye(w[i].shape[0])[:,:,np.newaxis,np.newaxis]
            wi_ = np.swapaxes(wi_,1,2).reshape(w[i].size,w[i].shape[0]*self.batch_size).transpose()

            w_.append(wi_)

            if i > 0:
                zi_ = -w[i][:,:-1].transpose()[:,:,np.newaxis]*ai
                zi_ = np.eye(self.batch_size)*zi_[:,:,:,np.newaxis]
                zi_ = np.swapaxes(zi_,1,2).reshape(z[i].size,z[i+1].size).transpose()
                z_.append(zi_)

        
        w_[-1] = w_[-1]/np.sqrt(beta)
        z_[-1] = z_[-1]/np.sqrt(beta)

        w_ = [sparse.csc_matrix(wi_) for wi_ in w_]
        z_ = [sparse.csc_matrix(zi_) for zi_ in z_]
        
        z_ = sparse.block_diag(z_)
        z_ = sparse.vstack((sparse.csr_matrix((w_[0].shape[0],z_.shape[1])),z_))
        z_ = z_+sparse.eye(z_.shape[0],z_.shape[1])
        w_ = sparse.block_diag(w_)
        J = sparse.hstack((w_,z_))
        return J
            

    def fit_alm(self,val_data=None,beta=10,tau=1e-2):
        l = np.random.normal(0,1,self.nz)
        u = self.write()
        sigma_0,h_0 = 1,la.norm(self.h(u))
        hist = {"tol":np.empty(0),"njev":np.empty(0),"loss":np.empty(0),'val_loss':np.empty(0)}
        for k in range(10):
            beta_k = np.power(beta,k)
            eta_k = 1/beta_k
            fun = lambda u: self.L(u,beta_k,l)
            jac = lambda u: self.JL(u,beta_k)
            try:
                sol = op.least_squares(fun,u,jac,ftol=None,xtol=None,gtol=eta_k,tr_solver='lsmr')
            except:
                print("Divide by 0")
                break
            u = sol.x

            w,_ = self.read(u)
            self.set_weights(w)

            h = self.h(u)

            sigma = sigma_0*np.amin([h_0*np.power(np.log(2),2)/la.norm(h)/(k+1)/np.power(np.log(k+2),2),
            																						 1])
            l = l + sigma*h

            # Convergence of Lagrangian function
            jac = self.JL(u,beta_k).toarray()
            tol = la.norm(2*self.JL(u,beta_k).transpose()*self.L(u,beta_k,l))+la.norm(h)
            hist['tol'] = np.append(hist['tol'],tol)
            hist['njev'] = np.append(hist['njev'],sol.njev)

            # Convergence of training loss
            y_pred = self.model(self.x)
            mse = keras.losses.MeanSquaredError()
            loss = mse(self.y,y_pred).numpy()
            hist['loss'] = np.append(hist['loss'],loss)
            if(val_data):
                # Convergence of val/test loss
                y_val_pred = self.model(val_data[0])
                val_loss = mse(y_val_pred,val_data[1]).numpy()
                print("epoch: ",k+1,"DL: ",tol,"njev: ",sol.njev, 
                	  "loss = ",loss,"val_loss = ",val_loss)
                hist['val_loss'] = np.append(hist['val_loss'],val_loss)

            if k>1 and ((1+tau)*hist['loss'][-1] > hist['loss'][-2]):
                break
        return hist

\end{verbatim}

\subsection{Numerical verification of Jacobian Matrix}
\label{AD}
\begin{verbatim}
# almtest.py
import net
from algopy import UTPM
import tensorflow as tf
import numpy as np
from tensorflow import keras
import alm
from matplotlib import pyplot

I,O,W,D,N, = 2,1,3,2,21
mu = 10

shape = (W,D,I,O)
sigma  = lambda x: np.tanh(x)
sigma_ = lambda x: 2/(np.cosh(2*x)+1)

tau  = lambda x: x
tau_ = lambda x: np.ones(x.shape)

x = np.random.normal(0,1,I*N).reshape((I,N))
y = np.random.normal(0,1,O*N).reshape((O,N))
u = np.random.random_sample((W*W*(D-1) + D*W + I*W + O*W + O + D*W*N,))
l = np.random.random_sample((D*W*N,))

nn = net.Net(shape,sigma,sigma_,tau,tau_,x,y)

_,z = nn.sim(u)
n_u = u.size
u[n_u-W*D*N:n_u] = z.ravel()

L = nn.eval_L(u,mu,l)
J = nn.eval_J_L(u,mu,l)

u_U = UTPM.init_jacobian(u)
L_U = net.eval_L_U(u_U,mu,l,nn)
J_U = UTPM.extract_jacobian(L_U)
J_U = J_U.reshape(J_U.shape[1:3])


model = keras.Sequential()
model.add(alm.Dense_d(activation_=sigma_,units=W,activation=tf.math.tanh,input_shape=(I,)))
model.add(alm.Dense_d(activation_=sigma_,units=W,activation=tf.math.tanh))
model.add(alm.Dense_d(activation_=tau_,units=O,activation=tau))
n2 = alm.ALMModel(model, x.transpose(), y.transpose())
model.summary()

g = np.arange(W)+1
a = g*(I+1)
b = a[2] + g*(W+1)
c = b[2] + W+1

g = np.r_[a,b,c]-1
h = np.arange(n_u-D*W*N,n_u)

mask = np.ones((n_u,),bool)

mask[g] = 0
mask[h] = 0

k = np.arange(n_u)
t = np.arange(n_u)
t[mask] = k[0:h[0]-D*W-O]
t[g] = k[h[0]-D*W-O:h[0]]
t[h] = k[h]

L2 = n2.L(u[t], mu, l)
J2 = n2.JL(u[t], mu).toarray()

Jnz = J2 != 0

J2 = np.c_[J2[:,mask],J2[:,g],J2[:,h]]

print(np.allclose(L,np.ravel(L_U.data[0,0])))
print(np.allclose(L,L2))
print(np.allclose(J,J_U))
print(np.allclose(J,J2))

np.set_printoptions(precision=3,linewidth=200)
pyplot.matshow(Jnz,cmap="binary")
pyplot.show()

\end{verbatim}

\section{batch tests}
\subsection{tanh test}
\begin{verbatim}
import net
from algopy import UTPM
import tensorflow as tf
import numpy as np
from tensorflow import keras
import alm
from matplotlib import pyplot
import time

rng = np.random.default_rng()
delta = 0.1
W = 8
K = 20

sigma  = lambda x: tf.math.tanh(x)
sigma_ = lambda x: 2/(np.cosh(2*x)+1)

tau  = lambda x: x
tau_ = lambda x: np.ones(x.shape)


for N in [10,20,40]:
    x = np.linspace(0,np.pi,N).reshape((N,1))
    y = np.sin(x*x)+rng.normal(0,delta,x.shape)
    per = rng.permutation(N)

    te = np.empty((K,4))
    for k in range(K):
        madam = keras.Sequential() 
        madam.add(keras.layers.Dense(activation="tanh",units=W,input_shape=(x.shape[1],)))
        madam.add(keras.layers.Dense(activation="tanh",units=W))
        madam.add(keras.layers.Dense(units=y.shape[1]))
        adam = keras.optimizers.Adam()
        es = keras.callbacks.EarlyStopping(monitor='loss',patience=10)
        madam.compile(optimizer=adam, loss='mean_squared_error')

        malm = keras.Sequential()
        malm.add(alm.Dense_d(activation=sigma,activation_=sigma_,units=W,input_shape=(x.shape[1],)))
        malm.add(alm.Dense_d(activation=sigma,activation_=sigma_,units=W))
        malm.add(alm.Dense_d(activation=tau,activation_=tau_,units=y.shape[1]))
    
        w = madam.get_weights()
        malm.set_weights(w)
        
        t0 = time.process_time()
        hist = madam.fit(x[per,:],y[per,:],batch_size=N,epochs=4000,callbacks=[es],verbose=0)
        t1 = time.process_time()-t0
        e1 = len(hist.history['loss'])
        print("t: ",t1,"k: ",e1,'loss: ', hist.history['loss'][-1])

        almnet = alm.ALMModel(malm, x[per,:], y[per,:])

        t0 = time.process_time()
        hist2 = almnet.fit_alm()
        t2 = time.process_time()-t0
        e2 = hist2['loss'].size
        print("t: ",t2,"k: ",e2, 'loss: ', hist2['loss'][-1])
    
        te[k,:] = [t1,t2,e1,e2]
        with open('testtanhn.npy','ab') as f:
            np.save(f,np.array(hist.history['loss']))
            np.save(f,np.concatenate([np.array(value) for value in 
                                      hist2.values()]).reshape((3,-1)).transpose())
    with open('testtanhn.npy','ab') as f:
        np.save(f,te)

\end{verbatim}
\subsection{relu test}
\begin{verbatim}
import net
from algopy import UTPM
import tensorflow as tf
import numpy as np
from tensorflow import keras
import alm
from matplotlib import pyplot
import time

rng = np.random.default_rng()
delta = 0.1
W = 16
K = 20

sigma  = lambda x: tf.nn.relu(x)
sigma_ = lambda x: np.greater(x,0,x)

tau  = lambda x: x
tau_ = lambda x: np.ones(x.shape)


for N in [20,40,80]:
    x = np.linspace(0,np.pi,N).reshape((N,1))
    y = np.sin(x*x)+rng.normal(0,delta,x.shape)
    per = rng.permutation(N)

    te = np.empty((K,4))
    for k in range(K):
        madam = keras.Sequential() 
        madam.add(keras.layers.Dense(activation="relu",units=W,input_shape=(x.shape[1],)))
        madam.add(keras.layers.Dense(activation="relu",units=W))
        madam.add(keras.layers.Dense(units=y.shape[1]))
        adam = keras.optimizers.Adam()
        es = keras.callbacks.EarlyStopping(monitor='loss',patience=10)
        madam.compile(optimizer=adam, loss='mean_squared_error')

        malm = keras.Sequential()
        malm.add(alm.Dense_d(activation=sigma,activation_=sigma_,units=W,input_shape=(x.shape[1],)))
        malm.add(alm.Dense_d(activation=sigma,activation_=sigma_,units=W))
        malm.add(alm.Dense_d(activation=tau,activation_=tau_,units=y.shape[1]))
    
        w = madam.get_weights()
        malm.set_weights(w)
        
        t0 = time.process_time()
        hist = madam.fit(x[per,:],y[per,:],batch_size=N,epochs=4000,callbacks=[es],verbose=0)
        t1 = time.process_time()-t0
        e1 = len(hist.history['loss'])
        print("t: ",t1,"k: ",e1,'loss: ', hist.history['loss'][-1])

        almnet = alm.ALMModel(malm, x[per,:], y[per,:])

        t0 = time.process_time()
        hist2 = almnet.fit_alm()
        t2 = time.process_time()-t0
        e2 = hist2['loss'].size
        print("t: ",t2,"k: ",e2, 'loss: ', hist2['loss'][-1])
    
        te[k,:] = [t1,t2,e1,e2]
        with open('testrelun.npy','ab') as f:
            np.save(f,np.array(hist.history['loss']))
            np.save(f,np.concatenate([np.array(value) for value in 
                                      hist2.values()]).reshape((3,-1)).transpose())
    with open('testrelun.npy','ab') as f:
        np.save(f,te)
\end{verbatim}



% ... en zo verder tot
\include{app-n}

\backmatter
% Na de bijlagen plaatst men nog de bibliografie.
% Je kan de  standaard "abbrv" bibliografiestijl vervangen door een andere.
\bibliographystyle{abbrv}
\bibliography{references}

\end{document}

%%% Local Variables: 
%%% mode: latex
%%% TeX-master: t
%%% End: 
